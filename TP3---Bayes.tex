% Options for packages loaded elsewhere
\PassOptionsToPackage{unicode}{hyperref}
\PassOptionsToPackage{hyphens}{url}
%
\documentclass[
]{article}
\usepackage{amsmath,amssymb}
\usepackage{iftex}
\ifPDFTeX
  \usepackage[T1]{fontenc}
  \usepackage[utf8]{inputenc}
  \usepackage{textcomp} % provide euro and other symbols
\else % if luatex or xetex
  \usepackage{unicode-math} % this also loads fontspec
  \defaultfontfeatures{Scale=MatchLowercase}
  \defaultfontfeatures[\rmfamily]{Ligatures=TeX,Scale=1}
\fi
\usepackage{lmodern}
\ifPDFTeX\else
  % xetex/luatex font selection
\fi
% Use upquote if available, for straight quotes in verbatim environments
\IfFileExists{upquote.sty}{\usepackage{upquote}}{}
\IfFileExists{microtype.sty}{% use microtype if available
  \usepackage[]{microtype}
  \UseMicrotypeSet[protrusion]{basicmath} % disable protrusion for tt fonts
}{}
\makeatletter
\@ifundefined{KOMAClassName}{% if non-KOMA class
  \IfFileExists{parskip.sty}{%
    \usepackage{parskip}
  }{% else
    \setlength{\parindent}{0pt}
    \setlength{\parskip}{6pt plus 2pt minus 1pt}}
}{% if KOMA class
  \KOMAoptions{parskip=half}}
\makeatother
\usepackage{xcolor}
\usepackage[margin=1in]{geometry}
\usepackage{graphicx}
\makeatletter
\def\maxwidth{\ifdim\Gin@nat@width>\linewidth\linewidth\else\Gin@nat@width\fi}
\def\maxheight{\ifdim\Gin@nat@height>\textheight\textheight\else\Gin@nat@height\fi}
\makeatother
% Scale images if necessary, so that they will not overflow the page
% margins by default, and it is still possible to overwrite the defaults
% using explicit options in \includegraphics[width, height, ...]{}
\setkeys{Gin}{width=\maxwidth,height=\maxheight,keepaspectratio}
% Set default figure placement to htbp
\makeatletter
\def\fps@figure{htbp}
\makeatother
\setlength{\emergencystretch}{3em} % prevent overfull lines
\providecommand{\tightlist}{%
  \setlength{\itemsep}{0pt}\setlength{\parskip}{0pt}}
\setcounter{secnumdepth}{-\maxdimen} % remove section numbering
\usepackage{booktabs}
\usepackage{longtable}
\usepackage{array}
\usepackage{multirow}
\usepackage{wrapfig}
\usepackage{float}
\usepackage{colortbl}
\usepackage{pdflscape}
\usepackage{tabu}
\usepackage{threeparttable}
\usepackage{threeparttablex}
\usepackage[normalem]{ulem}
\usepackage{makecell}
\usepackage{xcolor}
\ifLuaTeX
  \usepackage{selnolig}  % disable illegal ligatures
\fi
\IfFileExists{bookmark.sty}{\usepackage{bookmark}}{\usepackage{hyperref}}
\IfFileExists{xurl.sty}{\usepackage{xurl}}{} % add URL line breaks if available
\urlstyle{same}
\hypersetup{
  hidelinks,
  pdfcreator={LaTeX via pandoc}}

\title{\vspace{1cm}
\begin{tabular}{c}
{\normalsize\textbf{UNIVERSIDAD NACIONAL DE ROSARIO}}\\
{\Large Facultad de Ciencias Económicas y Estadística}\\
\\
\\
\includegraphics[width=5cm]{LogoUNR.png}\\
\\
\\
{\huge\textbf{Análisis de una masacre}}\\
{\huge\textbf{utilizando modelos lineales bayesianos}}\\
\\
{\Large Estadística Bayesiana - Trabajo Práctico Nº3}\\
\end{tabular}
\vspace{4cm}
\begin{flushright}
  \includegraphics[width=4cm]{chuky.png}
\end{flushright}}
\author{Alumnas: Agustina Mac Kay, Ailén Salas y Rocío Canteros}
\date{Año 2024}

\begin{document}
\maketitle

\hypertarget{introducciuxf3n}{%
\subsection{Introducción}\label{introducciuxf3n}}

La tanatocronología, derivada de las palabras griegas ``thanatos''
(muerte) y ``chronos'' (tiempo), es un subcampo de la medicina forense
que se centra en determinar el intervalo postmortem, es decir, el tiempo
transcurrido desde la muerte hasta el descubrimiento del
cadáver.\footnote{Fuente:
  \href{https://www.cun.es/diccionario-medico/terminos/tanatocronologia\#:~:text=La\%20tanatocronolog\%C3\%ADa\%2C\%20derivada\%20de\%20las,hasta\%20el\%20descubrimiento\%20del\%20cad\%C3\%A1ver.}{\textcolor{blue}{\underline{Diccionario Médico: tanatocronología - Clínica Universidad de Navarra}}}}

Desde el momento de la muerte, comienza en el cuerpo humano una serie de
procesos químicos y físicos que se conocen como fenómenos cadavéricos.
Uno de ellos es el enfriamiento del cuerpo (enfriamiento postmortem o
\emph{algor mortis}). En este proceso, la temperatura del cadáver
desciende hasta igualarse con la temperatura ambiente. Este descenso
ocurre más rápido en las primeras horas después de la muerte.

El objetivo de este trabajo es averiguar la hora de muerte de Sergio
Contreras, un hombre que fue hallado muerto desangrado, producto de una
herida punzante recibida en el bazo, en su casa en la localidad
cordobesa de Salsipuedes. Para resolver este acertijo, se cuenta con la
siguiente información sobre la temperatura del cuerpo sin vida del
hombre:

\begin{itemize}
\item
  5:33 hs - El Centro de Atención de Emergencias 911 recibe un llamado
  de Lidia Benegas, alertando sobre ruidos extraños en la vivienda de su
  vecino en la localidad de Salsipuedes.
\item
  6:00 hs - La policia arriba al lugar y se constata la presencia de un
  cuerpo tendido en el suelo.
\item
  6:45 hs - Arriba la policía científica al lugar del crimen, se
  determina la ausencia de signos vitales en el cuerpo de Sergio y el
  médico forense informa una temperatura corporal de 32.8 °C.
\item
  8:15hs - Finalizados los procedimientos legales y técnicos, se coloca
  el cuerpo en la bolsa de óbito para ser trasladado a morgue judicial.
  El termómetro registra que la temperatura del cadáver es de 30.5 °C.
\item
  13:30 hs - Comienza la autopsia. El cadáver se encuentra a 23.7 °C.
\end{itemize}

\hypertarget{secciuxf3n-con-nombre}{%
\subsection{Sección con nombre ????}\label{secciuxf3n-con-nombre}}

Como el ritmo con el cual el cuerpo pierde temperatura no es constante,
se puede pensar que la derivada de la temperatura respecto al tiempo
varía con el tiempo. En este caso, la temperatura del cadáver satisface
la siguiente ley:
\[ \frac{dT(t)}{dt} = r[T_{\text{amb}}-T(t)] \tag{1} \] donde
\(T_{\text{amb}}\) es la temperatura ambiente (un valor fijo y
conocido), \(r\) es una constante y \(T(t)\) es la función (por ahora
desconocida) que describe la temperatura del cuerpo en función del
tiempo.

Una posible función \(T(t)\) es:
\[ T(t)=T_{\text{amb}} + (T_i - T_{\text{amb}}) \cdot e^{-rt} \] siendo
\(T_i\) la temperatura a la que está inicialmente el cuerpo.

La misma satisface la ecuación \((1)\) ya que:
\[ \frac{dT(t)}{dt} = (T_i - T_{\text{amb}}) \cdot (-r) \cdot e^{-rt} = \]
\[ = -r \cdot [(T_i - T_{\mathrm{amb}}) \cdot e^{-rt}] = \]
\[ = -r \cdot [-T_{\mathrm{amb}}+T_{\mathrm{amb}}+(T_i - T_{\mathrm{amb}}) \cdot e^{-rt}] = \]
\[ = -r \cdot [-T_{\mathrm{amb}}+T(t)] = \]
\[ = r \cdot [T_{\mathrm{amb}} - T(t)] \] Se grafica a continuación la
función \(T(t)\) para distintos valores de la constante \(r\),
considerando una temperatura inicial del cuerpo de 37 Cº y una
temperatura ambiente de 23 Cº.

\includegraphics{TP3---Bayes_files/figure-latex/unnamed-chunk-2-1.pdf}

\emph{r} es una constante que representa la conductividad entre el
cuerpo y la superficie de contacto. A valores más altos de \emph{r}, más
rápido será el descenso de la temperatura corporal.

Los órganos abdominales pueden mantener el calor por al menos 24 horas.
En las primeras 12 horas se va perdiendo el calor de 0.8 a 1.0 grado
centígrado por hora y en las siguientes 12 horas de 0.3 a 0.5 grados
centígrados.\footnote{Fuente:
  \href{https://repository.unad.edu.co/bitstream/handle/10596/51040/vgomezh.pdf?sequence=3\&isAllowed=y\#:~:text=Enfriamiento\%20Cadav\%C3\%A9rico\%20o\%20Algor\%20Mortis,equilibra\%20con\%20el\%20medio\%20ambiente.}{\textcolor{blue}{\underline{Fenómenos cadavéricos - Valentina Gómez Hernández}}}}

Teniendo en cuenta esta información y lo observado en la Figura 1, se
puede considerar que un valor razonable para \emph{r} se encuentra entre
0.1 y 0.2.

\hypertarget{postulaciuxf3n-del-modelo}{%
\subsection{Postulación del modelo}\label{postulaciuxf3n-del-modelo}}

Para mayor simplicidad a la hora de construir un modelo, en lugar de
trabajar con la temperatura del cuerpo se decide utilizar la diferencia
entre la temperatura del mismo y la temperatura ambiente
\(T(t) - T_{\mathrm{amb}}\). Además, se define
\(T_{\mathrm{diff}} = T_i - T_{\mathrm{amb}}\); por lo que se obtiene
finalmente que
\(T(t) - T_{\mathrm{amb}} = T_{\mathrm{diff}} \cdot e ^ {-rt}\).

Se demuestra a continuación un resultado que será de vital importancia
para la construcción de un modelo: el logaritmo natural de la nueva
variable \(T(t) - T_{\mathrm{amb}}\) es una función lineal de \emph{t}.

\[ \ln(T(t) - T_{\mathrm{amb}}) = \ln( T_{\mathrm{diff}} \cdot e^{-rt}) = \ln(T_{\mathrm{diff}}) + (-rt) \cdot \ln (e) = \ln(T_{\mathrm{diff}}) - rt = \beta_{0} + \beta_{1}\cdot t \tag{2}\]

Se puede interpretar a \(\beta_{0}\) como el valor del
\(\ln(T(t) - T_{\mathrm{amb}})\) cuando \(t=0\). Al exponenciar ambos
miembros se obtiene que \[ T(t=0) - T_{\mathrm{amb}} = e^{\beta_0}\] Si
se toma como tiempo cero el momento de la primera medición, entonces la
temperatura del cuerpo en ese momento es igual a
\(e^{\beta_0} + T_{\mathrm{amb}}\)

\(\beta_1\) es el opuesto de la constante de conductividad. Representa
el cambio en el logaritmo de la diferencia entre la temperatura corporal
y la temperatura ambiente por cada hora transcurrida.

Para determinar la hora de muerte de Sergio, se trabajará con un modelo
lineal bayesiano; se determinarán creencias a priori para los parámetros
desconocidos para luego actualizarlas con las mediciones de temperatura
que se realizaron al cuerpo del hombre. Para este proceso, se considera
que la temperatura ambiente (constante a través del tiempo) es de 22ºC.

El modelo propuesto incompleto, porque faltan las distribuciónes a
priori de sus parámetros, es el siguiente:

\[
\begin{cases} 
\ln(T(t) - T_{\mathrm{amb}}) \sim\mathcal{N}_{(\mu, \sigma)} \\
\mu = \beta_{0} + \beta_{1}\cdot t
\end{cases}
\]

Se asignarán distribuciones a priori para los parámetros desconocidos
\(\beta_0\), \(\beta_1\) y \(\sigma\). Si bien \(\mu\) también es
desconocido, no se le asigna un prior propio ya que su valor depende
exclusivamente de \(\beta_0\) y \(\beta_1\), de forma que una vez
asignados valores para estos dos parámetros, \(\mu\) queda definido.

Como se menciona en la ecuación (2), \(\beta_0\) es el
\(\ln(T_{\mathrm{diff}}) = \ln(T_i - T_{\mathrm{amb}})\). Se sabe que la
temperatura ambiente es de 22ºC, mientras que la temperatura inicial del
cuerpo se encuentra entre 36,5ºC y 37,5ºC (temperatura habitual del
cuerpo humano). Con esto se puede decir que el valor mínimo para
\(\beta_0\) es \(\ln(36,5^\circ-22^\circ)=\ln(14,5^\circ)=2,67^\circ\),
y el máximo es \(\ln(37,5^\circ-22^\circ)= \ln(15,5^\circ)=2,74^\circ\).

Con este rango de valores, se puede suponer a priori una distribución
normal para \(\beta_0\), de media
\(\mu_{\beta_0} = \frac{2.74 + 2.67}{2}=2.705\) y desvío
\(\sigma_{\beta_0} = \frac{2.74-2.705}{3} = 0.012\), quedando así
definido su \emph{prior}.

\textbf{CAMBIO:} el tiempo 0 es el momento de la primer medición.
Sabemos que a las 5:33hs la vecina de Sergio escuchó ruidos provenientes
de su casa, y la primer medición de temperatura se hizo a las 6:45hs.
Entonces, podemos pensar que al momento de la primer medición Sergio
llevaba muerto entre 1 y 2 horas. En ese tiempo, de esperar que la
temperatura de su cuerpo haya descendido entre 2 y 6 grados, como mucho.
Si consideramos la temperatura normal del cuerpo humano en 36.5ºC,
entonces el \(\ln(T(t) - T_{\mathrm{amb}})\) máximo es
\(\ln((36.5-2) - 22) = 2.526\) y el mínimo es
\(\ln((36.5-6) - 22) = 2.14\) Entonces podemos decir que \(\beta_0\)
tiene a priori una distribución normal con media
\(\mu_{\beta_0} = \frac{2.526 + 2.14}{2}=2.333\) y desvío
\(\sigma_{\beta_0} = \frac{2.526-2.333}{3} = 0.064\)

Para definir una distribución para \(\beta_1\), se puede utilizar la
Figura 1; si bien el gráfico se realizó para una temperatura ambiente
1ºC más alta que la actual, las conclusiones son las mismas. Entonces,
si un valor razonable para \emph{r} se encuentra entre 0.1 y 0.2, se
puede decir que \(\beta_1 = -r\) toma valores entre -0.2 y -0.1.

Si nuevamente se utiliza una distribución normal, se obtiene que
\(\beta_1 \sim \mathcal{N}(\mu_{\beta_1},\sigma_{\beta_1})\), con media
\(\mu_{\beta_1} = \frac{-0.2-0.1}{2} = -0.15\) y desvío
\(\sigma_{\beta_1} = \frac{-0.1+0.15}{3} = 0.033\).

\textbf{CAMBIO:} sin embargo, como Sergio murió desangrado y la pérdida
de sangre puede acelerar el descenso de la temperatura corporal, se
permitirá que \emph{-r} tome valores entre -0.3 y -0.15. Por eso se
propone a priori una distribución normal de media
\(\mu_{\beta_1} = \frac{-0.3-0.15}{2} = -0.225\) y desvío
\(\sigma_{\beta_1} = \frac{-0.15+0.225}{3} = 0.025\).

Por último, \(sigma\) es el desvío del \(\ln(T(t) - T_{\mathrm{amb}})\).
La diferencia máxima entre la temperatura ambiente (22ºC) y \(T(t)\) se
da cuando \(T(t)\) es máxima (37,5º). En ese caso,
\(\ln(37.5-22)=2.74\). La diferencia mínima es de cero grados, y se da
cuando la temperatura del cuerpo alcanza la temperatura ambiente, pero
ahí el logaritmo no existe.

Como el desvío de una distribución es estrictamente positivo, se propone
para el mismo una distribución Half-normal, con desvío
\(\sigma_\sigma = \frac{2.74}{3} = 0.91\).

Por lo tanto, si se consideran independientes los parámetros entre sí,
el modelo con las distribuciones a priori propuestas es:

\[
\begin{cases}
\ln(T(t) - T_{\mathrm{amb}}) \sim\mathcal{N}_{(\mu, \sigma)} \\
\mu = \beta_{0} + \beta_{1}\cdot t \\
\\
\beta_0 \sim \mathcal{N}_{(2.705, 0.012)} \\
\beta_1 \sim \mathcal{N}_{(-0.15, 0.017)} \\
\sigma \sim \mathcal{N}^{+}_{(0.91)}
\end{cases}
\] Se realizan pruebas predictivas a priori para este modelo.

\includegraphics{TP3---Bayes_files/figure-latex/unnamed-chunk-3-1.pdf}

Podemos ver que el prior es bastante compatible con las observaciones.
También notamos que los 3 puntos no están alineados. Esto puede deberse
a que, si bien probamos que el logaritmo natural
\(T(t) - T_{\mathrm{amb}}\) es una función lineal de \emph{t}, estamos
suponiendo que la temperatura ambiente es fija cuando la última medición
se realizó habiendo trasladado el cuerpo, por lo que sería raro que la
temperatura ambiente realmente se mantenga.

\hypertarget{primer-observaciuxf3n}{%
\subsection{Primer observación}\label{primer-observaciuxf3n}}

Se actualiza el modelo con el primer dato observado: la temperatura del
cuerpo de Sergio en la primer medición (6:45hs) fue de 32.8ºC.

\[
\begin{array}{c|cc|cc}
\text{Parámetro} &\text{Media} &\text{desvío} & \hat{R} & \text{Número efectivo de muestras} \\
\hline
\beta_0 & 2.344 & 0.059 & 1.0009 & 2746 \\
\beta_1 & -0.20 & 0.033 & 1.0005 & 3174 \\
\sigma & 0.347 & 0.383 & 1.0016 & 3275 \\
\end{array}
\] Los 3 parámetros tienen buen \(\hat{R}\) y un número efectivo de
muestras muy alto.

Trace plots:

\includegraphics{TP3---Bayes_files/figure-latex/unnamed-chunk-5-1.pdf}

Las cadenas parece que convergen todas

Entonces, el modelo a posteriori con 1 observación es:

\[
\begin{cases}
\ln(T(t) - T_{\mathrm{amb}}) \sim\mathcal{N}_{(\mu, \sigma)} \\
\mu = \beta_{0} + \beta_{1}\cdot t \\
\\
\beta_0 \sim \mathcal{N}_{(2.344, 0.059)} \\
\beta_1 \sim \mathcal{N}_{(-0.2, 0.033)} \\
\sigma \sim \mathcal{N}^{+}_{(????)}
\end{cases}
\]

\includegraphics{TP3---Bayes_files/figure-latex/unnamed-chunk-6-1.pdf}

Según este nuevo modelo, Serio murió entre el tiempo -0.3 y 0.3.
Malísimo. Una sola observación no fue suficiente para ``bajar'' el b0

\hypertarget{modelo-con-2-obs}{%
\subsection{Modelo con 2 obs}\label{modelo-con-2-obs}}

\[
\begin{array}{c|cc|cc}
\text{Parámetro} &\text{Media} &\text{desvío} & \hat{R} & \text{Número efectivo de muestras} \\
\hline
\beta_0 & 2.362 & 0.048 & 1.005 & 1242 \\
\beta_1 & -0.16 & 0.031 & 1.006 & 1157 \\
\sigma & 0.162 & 0.214 & 1.004 & 1021 \\
\end{array}
\]

El número efectivo de muestras ya no es tan grande pero el \(\hat{R}\)
sigue siendo bueno.

todas mezcladas. eso es bueno. recorren bien la distri

La muestra parece ser buena Ahora, el modelo actualizado es:

\[
\begin{cases}
\ln(T(t) - T_{\mathrm{amb}}) \sim\mathcal{N}_{(\mu, \sigma)} \\
\mu = \beta_{0} + \beta_{1}\cdot t \\
\\
\beta_0 \sim \mathcal{N}_{(2.362, 0.048)} \\
\beta_1 \sim \mathcal{N}_{(-0.16, 0.031)} \\
\sigma \sim \mathcal{N}^{+}_{(????)}
\end{cases}
\]

Vemos posibles rectas con el nuevo modelo

\includegraphics{TP3---Bayes_files/figure-latex/unnamed-chunk-9-1.pdf}

Sergio se murió aprox entre\ldots{}

\hypertarget{modelo-con-3-obs}{%
\subsection{Modelo con 3 obs}\label{modelo-con-3-obs}}

\[
\begin{array}{c|cc|cc}
\text{Parámetro} &\text{Media} &\text{desvío} & \hat{R} & \text{Número efectivo de muestras} \\
\hline
\beta_0 & 2.333 & 0.061 & 1.001 & 2883 \\
\beta_1 & -0.23 & 0.029 & 1.008 & 1744 \\
\sigma & 0.334 & 0.232 & 1.006 & 1981 \\
\end{array}
\]

Los \(\hat{R}\) son todos buenos y los números efectivos de muestras
también.

todas mezcladas. eso es bueno. el sigma no esta tan bueno. RARI recorren
bien la distri

La muestra parece ser buena

Ahora, el modelo actualizado es:

\[
\begin{cases}
\ln(T(t) - T_{\mathrm{amb}}) \sim\mathcal{N}_{(\mu, \sigma)} \\
\mu = \beta_{0} + \beta_{1}\cdot t \\
\\
\beta_0 \sim \mathcal{N}_{(2.333, 0.061)} \\
\beta_1 \sim \mathcal{N}_{(-0.23, 0.029)} \\
\sigma \sim \mathcal{N}^{+}_{(???)}
\end{cases}
\]

\includegraphics{TP3---Bayes_files/figure-latex/unnamed-chunk-12-1.pdf}

Sergio se murió\ldots{}

\hypertarget{what-if-sergio-estaba-enfermo}{%
\subsection{¿What if\ldots{} Sergio estaba
enfermo?}\label{what-if-sergio-estaba-enfermo}}

Fue mencionado anteriormente que la pérdida de sangre por hemorragia
puede acelerar el enfriamiento del cadáver. Otro factor a considerar es
la presencia de enfermedades. Las enfermedades crónicas aceleran el
enfriamiento del cuerpo, mientras que otras, como la fiebre, lo
retardan.\footnote{Fuente:
  \href{chrome-extension://efaidnbmnnnibpcajpcglclefindmkaj/https://www.uv.es/gicf/3R1_Pen\%CC\%83a_GICF_31.pdf}{\textcolor{blue}{\underline{Fenómenos cadavéricos y el tanatocronodiagnóstico}}}}

Si Sergio hubiese tenido una enfermedad al momento de su muerte, se
deberían modificar las distribuciones a priori para \(\beta_1\) y
\(\beta_0\) de modo que sus medias sean menores.

\end{document}
